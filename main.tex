\documentclass[11pt,a4paper]{article}
\usepackage[colorinlistoftodos]{todonotes}
% THE FIGURE NUMBERS GOT MIXED UP BECAUSE WE STARTED MOVING THEM AROUND

% Suggestions for Holger:
% 1) Error plots (Figures 10, 11): smooth high-frequency errors, plot 9 lines on one plot (1d,10d,30d) * (3 C pools) for each site. This will result in 3 plots (1 for each site)
% Holger:
%   --> Do you mean figures 8,9 (relative error)? YES
%   --> 360-day moving average could do the smoothing: SOUNDS GOOD
%   --> 3 color x 3 line types = 9 lines per plot
% Holger 2:
%   --> Why would we actually like to smooth the error. Isn't it niche to see a wild noise around zero?



% 2) Create relative error figure for 14C analogous to new figures from #10 and #11
% Holger:
%   --> again: figures 8, 9?: YES, ANALOGOUS TO 8,9
%   --> only 14C  or also Delta 14C?: I THINK 14C
% Holger 2:
%   --> done for 1d and 10d




% 3) Delete Figures 8 and 9 (absolute error), they are duplicative of Figures 10, 11
% Holger:
%   --> figures 6 and 7 show absolute error
% Holger 2:
%   --> deleted figures with absolute error, kept relative error figures




% 4) Delete figures 6 and 7
% Holger:
%   --> I guess you mean figures 4 and 5?:YES
% Holger 2:
%   --> deleted



% 5) Figures 12 and 13 consolidated to one figure with 30 d time step (we're assuming the error for 30 d is still very small. If it's less than 1-10 percent, it's probably fine)
% Holger:
%   --> Do you mean 10 and 11 here (depth profiles of Delta 14C)?: YES
% Holger 2:
%   --> deleted 1d figure, kept 10d figure



% 6) Figures 14 and 15: consolidate to a single figure, column integrated, with 30 d time step. Add atmosphere 14C content. Make clear the difference between ELM and reconstruction using solid and dashed lines
% Holger:
%   --> 12, 13?: YES
%   --> At the moment these figures are column integrated.
%   --> Solid and dashed lines can hardly emphasize a difference, because they lie on top of each other. That's why I used dots for the discrete reconstruction.: BUT THE DOTS ARE NOT CLEAR. BIGGER?
%% Holger 2:
%   --> deleted 1d figure, kept 10d figure, increased dot size, added atmosphere, unified axes scales
%   --> filename: Delta_14C_through_time_per_pool.pdf




% 7) Figures 16 and 17: (depth resolved) remove these figures
% Holger:
%   --> 14, 15?: YES
%% Holger 2:
%   --> done


%%%%% for debugging %%%%%

\usepackage{xcolor}
\newcommand{\red}[1]{\textcolor{red}{#1}}
\newcommand{\blue}[1]{\textcolor{blue}{#1}}
\newcommand{\gray}[1]{\textcolor{gray}{#1}}

%%%%% true header %%%%%

\usepackage{authblk}
\usepackage{natbib}
\usepackage{graphicx}
\usepackage{booktabs}
\usepackage{amsmath,amssymb,amsfonts,amscd,amsthm}
\usepackage{bm}
\usepackage{pdflscape}
\usepackage[flushleft]{threeparttable}
\usepackage{lineno}
\usepackage{setspace}
\usepackage{textcomp}
\usepackage{times}
\usepackage{enumerate}

%%%%% personal definitions %%%%
% vectors and matrices
\renewcommand{\vec}[1]{\mathbf{#1}}
\newcommand{\tens}[1]{\mathrm{#1}}
\newcommand{\id}{\tens{Id}}

% integrals
\newcommand{\deriv}[1]{\frac{\mathrm{d}}{\mathrm{d}#1}}
\newcommand{\dd}[1]{\,\mathrm{d}#1}

% limits and sums
\newcommand{\intl}{\int\limits}
\newcommand{\suml}{\sum\limits}

% spaces
\newcommand{\R}{\mathbb{R}}



%%%%% title page data %%%%%

\title{\bf Inferring radiocarbon values in carbon cycle models from their numerical output}
\author[1]{Holger Metzler}
\author[2]{William Riley}
\author[2]{Qing Zhu}
\author[1]{Alison Hoyt}
\author[1]{Markus M\"uller}
\author[1]{Carlos A. Sierra}
\affil[1]{\it \small Max Planck Institute for Biogeochemistry, Hans-Kn\"oll-Str. 10, 07745 Jena, Germany}
\affil[2]{\it Climate and Ecosystem Sciences Division, Lawrence Berkeley National Laboratory, Berkeley 94720, USA}

\date{}

\begin{document}
\doublespace
\maketitle

\noindent
{\bf Running head}: Radiocarbon from model output

\vspace{2em}

\noindent
\textbf{Corresponding author}: \\ Carlos A. Sierra, Max-Planck-Institute for Biogeochemistry, Hans-Kn\"{o}ll-Str. 10, 07745 Jena, Germany. Phone: +49 3641 576133, fax: +49 3641 577100, email: csierra@bgc-jena.mpg.de.

\vspace{2em}

\noindent
{\bf Keywords}: carbon cycle models, carbon stocks, carbon fluxes, reservoir theory, compartment models, radiocarbon, dynamical systems,  model diagnostics

\vspace{2em}

\noindent
{\bf Type of paper}: Technical Advance; possible journals: JAMES, GMD
\newpage
\linenumbers

\begin{abstract}
Radiocarbon is a powerful tracer of the global carbon cycle that is commonly used to assess rates of carbon cycling in different Earth system reservoirs, and as benchmark to assess model performance. It is increasingly recognized that radiocarbon should be included in Earth system models, and it has been recommended that models for the sixth model inter-comparison project CMIP6 report the predicted radiocarbon values for a multitude of carbon pools. However, the inclusion of radiocarbon in many models puts an additional burden on model developers that may not have the time or resources for a detailed representation of radiocarbon dynamics. Here, we present an alternative approach that consists on computing radiocarbon values for carbon pools from the numerical output of the model. It requires the use of the computed stocks and fluxes among all carbon pools for a particular simulation of the model. From this output, it is possible to compute a time-dependent linear compartmental system with its respective state transition matrix. Using reported values of atmospheric radiocarbon as inputs to the system, the state transition matrix is then applied to compute radiocarbon values for each individual pool, the average value for the entire system, and the radiocarbon in the output flux. 
%These quantities are compared with pool ages, system ages, and transit times of carbon for the particular simulation being analyzed. 
We demonstrate the approach for the soil component of ELMv1-ECA, a vertically resolved model that already explicitly includes radiocarbon and includes seven pools and ten vertical layers. Results from our proposed method are nearly identical to the predictions of ELMv1-ECA, which shows the great potential for using this approach in not only CMIP6 models that do not include radiocarbon, but also model output from previous model inter-comparison exercises. 
\end{abstract}

\newpage

\section*{Introduction}
%\begin{itemize}
%\item Radiocarbon as a tracer of the global carbon cycle
%\item Recommendation to include radiocarbon in CMIP6 models
%\item \gray{Introduce approach developed by Metzler et al. to compute age and transit times in time-dependent nonlinear models}
%\item Introduce aims of the paper and organization
%\end{itemize}

Soil is an important global C reservoir, but there are still large uncertainties in the controls on soil C cycling, and its representation in models \citep{Luo2016}. The dynamics of radiocarbon in terrestrial ecosystems can be used to interpret the rates at which carbon is exchanged with the atmosphere and moves through vegetation and soils. As a result, radiocarbon has been used in many empirical studies to evaluate rates of soil C cycling. However, it has only more recently been used as a tool for model intercomparison \citep{He2016}. 

Measurements of radiocarbon offer great potential to constrain temporal dynamics of ecosystem models and assess their performance. However, to date, such comparisons have been limited to a small number of models which simulate C isotopes \citep{Koven2013, Tifafi2018}, as the majority of models do not explicitly represent isotope dynamics. The current coupled climate-carbon cycle model intercomparison project (C4MIP) of the 6th phase coupled model intercomparison project (CMIP6), requests model participants to explicitly model radiocarbon and include the corresponding output from the major land and ocean model compartments \citep{Jones2016}. However to date, many models do not represent radiocarbon, as it can be time-consuming to implement, and computationally-intensive to run. Thus, the goal of this work is to develop a tool to make radiocarbon-based comparisons of C cycling comparable across a wider suite of models.

The availability of radiocarbon output from a range of models would allow not only an additional data constraint against global datasets \citep{He2016, Mathieu2015}, it may also offer a new tool for differentiating patterns of C cycling across different model structures \citep{Sierra2014}. 
%There are two broad classes of models currently being applied for studies of the global carbon cycle: (1) pseudo-first order representations, based on concepts first described for Century (Parton et al. 1996) and RoTHC (cite) and (2) microbe and mineral-surface explicity models (eg Ahrens et al. 20XX, Dwivedi et al. 2017; Riley et al. 2014; Sulman et al 2018) that attempt to more explicitly represent the underlying mechanisms responsible for soil carbon transformations. 
% I removed this sentence because the classification portrayed here is too simple. 
There are literally hundreds of different models of soil carbon cycling proposed in the literature \citep{Manzoni2009SBB}, each portraying different hypotheses about internal system dynamics, with specific model structures and parameterizations. Broad generalizations of models have been proposed based on basic assumptions of soil organic matter dynamics \citep{Sierra2015}.
Computing radiocarbon output from these different classes of models, without the need to incorporate isotopes into the full model structure, would allow more widespread data-model comparison, and an enhanced ability to evaluate differences in the way soil C cycles in these models.

Linear response theory suggests that the impulse response function of a particular model can be used to compute isotope dynamics without explicitly simulating the isotope within the model, but rather using the response function to convolve the isotope tracer \citep{Thompson1999}. This means that it may not be necessary to explicitly represent isotope dynamics a priori in a model, but it can be obtained a posteriori through the convolution procedure. However, linear response theory relies on assumptions of linearity and steady-state of the model, which are not met for the type of transient simulations to be submitted to C4MIP. 

\citet{Metzler2018PNAS} recently developed a method to obtain a mathematical object, the state transition matrix (see details below), that generalizes response function theory for nonlinear models out of equilibrium. This approach removes previous assumptions that prevented the use of the linear response approach for the computation of isotope and tracer dynamics a posteriori from simulations. Therefore, it offers an opportunity to compute isotope dynamics from models that do not explicitly represent it, and could be applied to models from CMIP6. Since output from these MIPs is presented in discrete time steps, it is still a challenge to translate the state transition matrix approach from the continuous to the discrete case.

We present an approach to compute radiocarbon values for models which do not explicitly represent radiocarbon. To do this, we extend the approach presented in Metzler et al. (2018) for the computation of the state transition matrix in continuous time to the discrete time case, and apply it to reconstruct tracer and isotope dynamics from model output. We first give a mathematical overview of the approach and then present examples of its use. We test our approach by comparing our computed values to output from the ELMv1-ECA model, an Earth system model that explicitly represents radiocarbon. We successfully match the modeled radiocarbon values along the entire vertically resolved soil profile, across soil and litter pools over time. This is a valuable new framework with large potential for application to the CMIP archives and to models with more diverse structures. 

\section{Methods}
%\subsection{section*{From a carbon cycle model to a radiocarbon model}
%\begin{itemize}
%    \item Introduce the equation to compute radiocarbon
%    \item show two-pool example: explicit radiocarbon model vs. radiocarbon model coming from new formula
%\end{itemize}



\subsection*{Carbon cycle models as compartmental systems}
    Carbon cycle models are subject to the law of mass conservation and therefore can be represented by compartmental systems \red{(Anderson 1983, Jacquez 1993 SIAM)}.
    A compartmental system with $d\geq1$ compartments is of the general form of the initial value problem
    
    \red{Bill: what about vertical structure and transport? Shouldn't we introduce that here also?}
    \todo[inline, color=green!40]{I will introduce here more explicitly how to do that}
    \begin{equation}\label{eqn:CS_td_nonlin}
        \begin{aligned}
            \deriv{t}\,\vec{C}(t) &= \tens{B}(\vec{C}(t),t)\,\vec{C}(t) + \vec{u}(\vec{C}(t),t),\quad t>t_0,\\
            \vec{C}(t_0) &= \vec{C}^0.
        \end{aligned}
    \end{equation}
Here, $\vec{C}(t)$ is the vector of compartment contents at time $t$, $\tens{B}$ is a matrix-valued function that governs the cycling of carbon through the system, $\vec{u}(t)$ is the vector of external carbon inputs to the system, $t_0$ is some fixed initial time, and $\vec{C}^0$ is a given vector  of initial system contents. Note that the internal cycling as well as the external inputs can each depend on time (e.g. due to changing environmental conditions) and on the current carbon content of the system itself. Consequently, system \eqref{eqn:CS_td_nonlin} is time-dependent and nonlinear.
    
    To guarantee mass conservation, the matrix $\tens{B}(\vec{x},t)$ is required to be compartmental \todo{can we clarify what 'compartmental' means? above you say mass conservation defines being compartmental, so this sentence seems tautological?} \todo[color=green!40]{the definition of the matrix being compartmental is the following list of (i), (ii), (iii)} for all nonnegative $\vec{x}$ and $t>t_0$.
    Hence, $\tens{B}(\vec{x},t)$ satisfies the properties
    \begin{enumerate}[(i)]
        \item $B_{ii}(\vec{x},t)\leq0$ for all $i$,
        \item $B_{ij}(\vec{x},t)\geq0$ for all $i\neq j$,
        \item $\suml_{i=1}^d B_{ij}(\vec{x},t)\leq0$ for all $j$.
    \end{enumerate}
    The diagonal values of $B_{ii}$ can be interpreted as the cycling rate of compartment $i$, the off-diagonal entries $B_{ij}$ represent the rate of carbon flux from compartment $j$ to compartment $i$, and $z_j=-\sum_{i=1}^d B_{ij}$ is the rate of carbon leaving the system through pool $j$.
    
    We now assume that we know a unique (numerical) solution $\vec{C}=\vec{C}(t)$ of system \eqref{eqn:CS_td_nonlin}.
    Plugging this solution into the system, we obtain a new compartmental system
    \begin{equation}\label{eqn:CS_td_lin}
        \begin{aligned}
            \deriv{t}\,\widetilde{\vec{C}}(t) &= \widetilde{\tens{B}}(t)\,\widetilde{\vec{C}}(t) + \widetilde{\vec{u}}(t),\quad t>t_0,\\
            \widetilde{\vec{C}}(t_0) &= \vec{C}^0,
        \end{aligned}
    \end{equation}
    where $\widetilde{\tens{B}}(t)=\tens{B}(\vec{C}(t),t)$ and $\widetilde{\vec{u}}(t)=\vec{u}(\vec{C}(t),t)$.
    This system is still time-dependent, but not nonlinear anymore.
    Under the assumption that it has a unique solution $\widetilde{\vec{C}}=\widetilde{\vec{C}}(t)$, the two solution trajectories $\widetilde{\vec{C}}$ and $\vec{C}$ are identical.
    Then the linear compartmental system \eqref{eqn:CS_td_lin} describes exactly the same solution trajectory as the nonlinear compartmental system \eqref{eqn:CS_td_nonlin}.
    For the sake of simplicity of notation we omit the tilde in Eq. \eqref{eqn:CS_td_lin} from now on.
    
    The unique semi-analytical solution of linear compartmental system \eqref{eqn:CS_td_lin} is given by \red{(Brockett 2015)}
    \begin{equation*}
        \vec{C}(t) = \tens{\Phi}(t,t_0)\,\vec{C}^0 + \intl_0^t \tens{\Phi}(t,\tau)\,\vec{u}(\tau)\dd{\tau}.
    \end{equation*}
    We call the solution semi-analytic because the system's state transition matrix $\tens{\Phi}$ is only implicitly given as the unique solution of the matrix initial value problem
    \begin{equation*}
        \begin{aligned}
            \deriv{t}\,\tens{\Phi}(t,t_0) &= \tens{B}(t)\,\tens{\Phi}(t,t_0),\quad t>t_0,\\
            \tens{\Phi}(t_0,t_0) &= \id,
        \end{aligned}
    \end{equation*}
    where $\id$ denotes the identity matrix in $\R^d$. \todo{then why not just use that symbol?} \todo[color=green!40]{$\R^d$ is the $d$-dimensional space on which the identity matrix $\id$ acts., i.e. $\id:\R^d\to\R^d$ by $\id x=x$ for $x\in\R^d$.}

\subsection*{The radiocarbon compartmental system}
    Once we know the representation \eqref{eqn:CS_td_nonlin} of a carbon cycle model, we can readily manipulate the equation to obtain a compartmental system for radiocarbon.
    We only need to substitute $\tens{B}$ by ${}^{14}\tens{B}$ defined by
    \begin{equation*}
        {}^{14}\tens{B}(\vec{C}(t),t) = \tens{B}(\vec{C}(t),t) - \lambda\,\id,
    \end{equation*}
    where $\lambda$ is the radioactive decay constant of ${}^{14}$C scaled to the right time unit.
    For example, if the system is observed in time units of years, then $\lambda=\ln(2)/5568$.
    This manipulation leads to a state transition matrix of the radiocarbon system given by
    \begin{equation*}
        {}^{14}\tens{\Phi}(t,t_0) = \tens{\Phi}(t,t_0)\, e^{-\lambda\,(t-t_0)}\,\id, \quad t\geq t_0,
    \end{equation*}
    from which we can directly see the effect of the radioactive decay.


\section*{Approach to reconstruct a carbon cycle model from data (Holger, Markus)}
%\begin{itemize}
%    \item global carbon cycle models as compartmental systems
%    \item Show how to compute a time-dependent compartmental model from fluxes and stocks
%    \item \gray{Introduce the state transition matrix and explain what it does}
%\end{itemize}


\subsection*{Reconstruction of a global carbon cycle model from discrete-time model output}
    \red{some introducing text like: the above approach works only when we have the model completely described by an ODE, unfortunately, very often we only have model output at discrete points in time $\to$ model reconstruction necessary as a first step towards radiocarbon}
    
    \subsubsection*{One-dimensional case}
        Suppose we are given a one-dimensional \todo{1D usually refers to space. I guess you mean only time-dependent? Is this still one compartment? Is that what you mean by 1D?} \todo[color=green!40]{the dimension is supposed to refer to the space} time-dependent linear compartmental system
        \begin{equation}\label{eqn:CS_one_dim}
            \begin{aligned}
                \deriv{t}\,C(t) &= \gamma(t)\,C(t) + u(t),\quad t>t_0,\\
                C(t_0) &= C^0,
            \end{aligned}
        \end{equation}
        of which $\gamma(t)$ and $u(t)$ are unknown.
        But for discrete times $t_0<t_1<t_2<\cdots<t_n$ we are given
        \begin{itemize}
            \item the initial system content $C^0$,
            \item $u^k = \intl_{t_k}^{t_{k+1}} u(t)\dd{t},\quad k=0,1,\ldots,n-1$, and
            \item $r^k = \intl_{t_k}^{t_{k+1}} r(t)\dd{t},\quad k=0,1,\ldots,n-1$.
        \end{itemize}
        Here, $u^k$ and $r^k$ denote the accumulated external system input and the accumulated external system output in the time interval $I_k=[t_k,t_{k+1}]$, $k=0,1,\ldots,n-1$, respectively.

        We are interested in constructing a compartmental system, as simple as possible, that matches these given data as well as possible.
        To that end, we approximate the (unknown) system \eqref{eqn:CS_one_dim} on each interval $I_k$ by a linear time-independent system
        \begin{equation}\label{eqn:CS_one_dim_approx}
            \begin{aligned}
                \deriv{t}\,\widehat{C}(t) &= \widehat{\gamma}^k\,\widehat{C}(t) + \widehat{u}^k,\quad t\in I_k,\\
                \widehat{C}(t_k) &= \widehat{C}^k.
            \end{aligned}
        \end{equation}
        Suppose that at time step $k$ the value $\widehat{C}^k$ is already known from the previous time step.
        This is no restriction, because for $k=0$ we set $\widehat{C}^0=C^0$.
        
        We write $\Delta_k=t_{k+1}-t_k$ and choose $\widehat{u}^k=u^k/\Delta_k$.
        Consequently, for the time interval $I_k$ the cumulative external system input $u^k$ of the unknown system \eqref{eqn:CS_one_dim} and the cumulative external system input $\Delta_k\,\widehat{u}^k$ of the approximating system \eqref{eqn:CS_one_dim_approx} coincide.
        We are left with finding good approximations $\widehat{\gamma}^k$ for $\gamma(t)$ on $I_k$, $k=0,1,\ldots,n-1$.
        We can compute the cumulative external system output of the approximating system \eqref{eqn:CS_one_dim_approx} on $I_k$, in dependence on the choice of $\widehat{\gamma}^k$, by
        \begin{equation*}
            \widehat{r}^k(\widehat{\gamma}^k) = -\widehat{\gamma}^k\intl_{t_k}^{t_{k+1}} \widehat{C}(t)\dd{t},
        \end{equation*}
        where, for $t\in I_k$,
        \begin{align*}
            \widehat{C}(t) &= e^{-\widehat{\gamma}^k\,(t-t_k)}\,\widehat{C}^k + \intl_{t_k}^t e^{-\widehat{\gamma}^k\,(t-\tau)}\,\widehat{u}^k\dd{\tau}\\
            &= e^{-\widehat{\gamma}^k\,(t-t_k)}\,\widehat{C}^k + (\widehat{\gamma}^k)^{-1}\,(e^{-\widehat{\gamma}^k\,(t-t_k)}-1)\,\widehat{u}^k
        \end{align*}
        is the unique solution of \eqref{eqn:CS_one_dim_approx} on $I_k$.
        Consequently, we obtain $\widehat{\gamma}^k$ by solving the one-dimensional nonlinear optimization problem
        \begin{equation*}
            \underset{\widehat{\gamma}^k<0}{\operatorname{minimize}}\,|\widehat{r}^k(\widehat{\gamma}^k)-r^k|.
        \end{equation*}
        We define $\widehat{C}^{k+1}=\widehat{C}(t_{k+1})$ and continue with the next time step $k+1$.
        
        In the special case of a time-independent unknown system \eqref{eqn:CS_one_dim}, i.e. $\gamma(t)=\gamma<0$ and $u(t)=u\geq0$, the approximation by system \eqref{eqn:CS_one_dim_approx} is perfect, independent of the number $n$ of given data points.\\
        
    \subsubsection*{Mutli-dimensional case}
        If the compartmental system consists of more than one compartment, internal cycling among the different compartments may occur. As described below, all modern soil carbon models are multi-compartmental, and they often represent vertical structure with explicit representation of diffusive or advective fluxes that connect adjacent soil layers in the discretized model.
        An approximating system is then required to also match the internal fluxes.
        Suppose we are given a $d$-dimensional time-dependent linear compartmental system
%        \begin{equation}\label{eqn:CS_two_dim}
%            \begin{aligned}
%                \deriv{t}\,\begin{pmatrix} C_1 \\ C_2 \end{pmatrix}(t) &= 
%                \begin{pmatrix} B_{11} & B_{12} \\ B_{21} & B_{22} \end{pmatrix}(t)\,
%                \begin{pmatrix} C_1 \\ C_2 \end{pmatrix}(t) + 
%                \begin{pmatrix} u_1 \\ u_2 \end{pmatrix}(t),\quad t>t_0,\\
%                \begin{pmatrix} C_1 \\ C_2 \end{pmatrix}(t_0) &=
%                \begin{pmatrix} C^0_1 \\ C^0_2 \end{pmatrix},
%            \end{aligned}
%        \end{equation}
        \begin{equation}\label{eqn:CS_multi_dim}
            \begin{aligned}
                \deriv{t}\,\vec{C}(t) &= \tens{B}(t)\,\vec{C}(t) + \vec{u}(t),\quad t>t_0,\\
                \vec{C}(t_0) &= \vec{C}^0,
            \end{aligned}
        \end{equation}
        of which $\tens{B}(t)=(B_{ij}(t))_{i,j=1,2,\ldots,d}$ and $\vec{u}(t)=(u_i(t))_{i=1,2,\ldots,d}$ are unknown.
        But for discrete times $t_0<t_1<t_2<\cdots<t_n$ we are given
        \begin{itemize}
            \item the initial system content vector $\vec{C}^0=(C^0_1,C^0_2,\ldots,C^0_d)^T$,
            \item $\vec{u}^k = \intl_{t_k}^{t_{k+1}} \vec{u}(t)\dd{t},\quad k=0,1,\ldots,n-1$, 
            \item $\vec{r}^k = \intl_{t_k}^{t_{k+1}} \vec{r}(t)\dd{t},\quad k=0,1,\ldots,n-1$, and
            \item $F_{ij}^k = \intl_{t_k}^{t_{k+1}} B_{ij}(t)\,C_j(t)\dd{t},\quad i\neq j$.
        \end{itemize}
        Here, $\vec{u}^k$ and $\vec{r}^k$ denote the accumulated external system input vector and the accumulated external system output vector in the time interval $I_k=[t_k,t_{k+1}]$, $k=0,1,\ldots,n-1$, respectively.
        Furthermore, $F^k_{ij}$ denotes the accumulated flux from compartment $j$ to compartment $i$ during $I_k$.

        We are interested in constructing a compartmental system, as simple as possible, that matches these given data as well as possible.
        To that end, we approximate the (unknown) system \eqref{eqn:CS_multi_dim} on each interval $I_k$ by a linear time-independent system
        \todo{Why do you call this time-independent, when time is in the equation?} \todo[color=green!40]{because $\tens{B}$ and $\vec{u}$ do not depend on time $t$, then this is the usual terminology: time-independent or autonomous}
        \begin{equation}\label{eqn:CS_multi_dim_approx}
            \begin{aligned}
                \deriv{t}\,\widehat{C}(t) &= \widehat{\tens{B}}^k\,\widehat{C}(t) + \widehat{\vec{u}}^k,\quad t\in I_k,\\
                \widehat{\vec{C}}(t_k) &= \widehat{\vec{C}}^k.
            \end{aligned}
        \end{equation}
        Suppose that at time step $k$ the vector $\widehat{\vec{C}}^k$ is already known from the previous time step.
        This is no restriction, because for $k=0$ we set $\widehat{\vec{C}}^0=\vec{C}^0$.

        Again, we write $\Delta_k=t_{k+1}-t_k$ and choose $\widehat{\vec{u}}^k=\vec{u}^k/\Delta_k$.
        Consequently, for the time interval $I_k$ the cumulative external system input vector $\vec{u}^k$ of the unknown system \eqref{eqn:CS_multi_dim} and the cumulative external system input vector $\Delta_k\,\widehat{\vec{u}}^k$ of the approximating system \eqref{eqn:CS_multi_dim_approx} coincide.
        We are left with finding good approximations $\widehat{\tens{B}}^k$ for $\tens{B}(t)$ on $I_k$, $k=0,1,\ldots,n-1$ under the constraint that $\widehat{\tens{B}}^k$ is a compartmental matrix.
        To that end, we try to match the internal fluxes $F^k_{ij}$ and the external output flux vectors $\vec{r}^k$ at well as possible.
        For $i\neq j$, we can compute the internal flux from $j$ to $i$ of the approximating system \eqref{eqn:CS_multi_dim_approx}, in dependence on the choice of $\widehat{\tens{B}}^k$, during the time interval $I_k$ by
        \begin{equation*}
            \widehat{F}^k_{ij}(\widehat{\tens{B}}^k) = \widehat{B}^k_{ij}\intl_{t_k}^{t_{k+1}} \widehat{C}_j(t)\dd{t},
        \end{equation*}
        where, for $t\in I_k$,
        \begin{align*}
            \widehat{\vec{C}}(t) &= e^{(t-t_k)\,\widehat{\tens{B}}^k}\,\widehat{\vec{C}}^k + \intl_{t_k}^t e^{(t-\tau)\,\widehat{\tens{B}}^k}\,\widehat{\vec{u}}^k\dd{\tau}\\
            &= e^{(t-t_k)\,\widehat{\tens{B}}^k}\,\widehat{\vec{C}}^k + (\widehat{\tens{B}}^k)^{-1}\,(e^{(t-t_k)\,\widehat{\tens{B}}^k})-\id)\,\widehat{\vec{u}}^k
        \end{align*}
        is the unique solution of \eqref{eqn:CS_multi_dim_approx} on $I_k$, and $e^{(t-t_k)\,\widehat{\tens{B}}^k}$ denotes the matrix exponential.

        Furthermore, we define by $\widehat{z}^k_j=-\sum_{i=1}^d \widehat{B}^k_{ij}>0$ the rate of external outflow from compartment $j$.
        Then we can compute the cumulative external system output through compartment $j$ of the approximating system \eqref{eqn:CS_multi_dim_approx} during $I_k$, in dependence on the choice of $\widehat{\tens{B}}^k$, by
        \begin{equation*}
            \widehat{r}^k_j(\widehat{\tens{B}}^k) = \widehat{z}^k_j\intl_{t_k}^{t_{k+1}} \widehat{C}^k_j(t)\dd{t}.
        \end{equation*}
        Consequently, we obtain $\widehat{\tens{B}}^k$ by solving the $m$-dimensional nonlinear optimization problem
        \begin{equation*}
            \underset{\widehat{\tens{B}}^k\text{ compartmental}}{\operatorname{minimize}}\,\left[\max\limits_{i\neq j} |\widehat{F}^k_{ij}(\widehat{\tens{B}}^k)-F^k_{ij}| + |\widehat{\vec{r}}^k(\widehat{\tens{B}}^k)-\vec{r}^k|\right].
        \end{equation*}
        The dimension $m$ of the optimization problem is at maximum $d^2$.
        If some internal fluxes or external output fluxes are inexistent, then $m<d^2$.
        We define $\widehat{\vec{C}}^{k+1}=\widehat{\vec{C}}(t_{k+1})$ and continue with the next time step $k+1$.\\
       
\subsection*{Discrete time version}
    \begin{itemize}
        \item why discrete time?
        \item general solution formula, state transition matrix in discrete shape
        \item differences to continuous-time reconstruction
        \item Use two-pool example from above to show issues of approximation of the exact solution: interpolation between time-steps in the continuous approach and time-step size in the discrete approach
        \item[$\to$] might lead to discrete model with variable time step
        \item Discuss issues of computation time and tolerance.
    \end{itemize}
    
    Model output is usually provided at discrete points $t_0<t_1<\cdots<t_n$ in time, either because there is no feasible way to represent continuous-time model output or the model itself works on basis of discrete time steps.
    In either case, if we reconstruct the unknown original model in terms of a continuous-time approximation \eqref{eqn:CS_multi_dim_approx}, we interpolate the given date for points in time that lie between $t_k$ and $t_{k+1}$.
    This interpolation can of course be done in many different and arbitrary ways.
    We can get rid of this sponginess by sticking to the exact data that are given.
    Consequently, we reconstruct the unknown model in a discrete time setting, on the time grid $t_0<t_1<\cdots<t_n$ that is predetermined by the given data and make no guesses for any other point in time.
    This leads to a reconstructed model which is represented by a $d$-dimensional discrete-time compartmental system of shape
    \begin{equation}\label{eqn:CS_discrete}
        \begin{aligned}
            \widehat{\vec{C}}(k+1) &= \widehat{\tens{B}}(k)\,\widehat{\vec{C}}(k) + \widehat{\vec{u}}(k),\quad k=0,1,\ldots,n-1,\\
            \widehat{\vec{C}}(0) &= \widehat{\vec{C}}^0.
        \end{aligned}
    \end{equation}
    The discrete-time compartmental matrices $\widehat{\tens{B}}(k)$ satisfy for all $k=0,1,\ldots,n-1$ the two conditions
    \begin{enumerate}[(i)]
        \item $\widehat{B}_{ij}(k)\geq0$ for all $i,j=1,2,\ldots,d$, and
        \item $\suml_{i=1}^d \widehat{B}_{ij}(k)\leq 1$ for all $j=1,2,\ldots,d$.
    \end{enumerate}
    Analogously to the continuous-time case, $\widehat{z}_j(k)=1-\sum_{i=1}^d \widehat{B}_{ij}(k)$ is the rate at which mass at time step $k$ leaves the system through compartment $j$.
    The discrete-time state transition matrix is, for $0\leq k_1<k_2\leq n$, given by
    \begin{equation*}
        \widehat{\tens{\Phi}}(k_2,k_1) = \widehat{\tens{B}}_{k_2-1}\cdot\widehat{\tens{B}}_{k_2-2}\cdots\widehat{\tens{B}}_{k_1+1}\cdot\widehat{\tens{B}}_{k_1} = \prod\limits_{k=k_1}^{k_2-1}\widehat{\tens{B}}_k,
    \end{equation*}
    and $\widehat{\tens{\Phi}}(k,k)=\id$.
    The solution of the discrete-time initial value problem \eqref{eqn:CS_discrete} is then given by
    \begin{equation*}
        \widehat{\vec{C}}(k) = \widehat{\tens{\Phi}}(k,0)\,\widehat{\vec{C}}^0 + \suml_{m=0}^{k-1} \widehat{\tens{\Phi}}(k,m)\,\widehat{\vec{u}}(m),\quad k=0,1,\ldots,n.
    \end{equation*}
    
    Suppose we are given a $d$-dimensional time-dependent linear compartmental system with structure \eqref{eqn:CS_multi_dim} and unknown $\tens{B}$ and $\tens{u}$.
    For discrete times $t_0<t_1<t_2<\cdots<t_n$ we are given
    \begin{itemize}
        \item the initial system content vector $\vec{C}^0=(C^0_1,C^0_2,\ldots,C^0_d)^T$,
        \item $\vec{u}^k = \intl_{t_k}^{t_{k+1}} \vec{u}(t)\dd{t},\quad k=0,1,\ldots,n-1$, 
        \item $\vec{r}^k = \intl_{t_k}^{t_{k+1}} \vec{r}(t)\dd{t},\quad k=0,1,\ldots,n-1$, and
        \item $F_{ij}^k = \intl_{t_k}^{t_{k+1}} B_{ij}(t)\,C_j(t)\dd{t},\quad i\neq j$.
    \end{itemize}
    Here, $\vec{u}^k$ and $\vec{r}^k$ denote the accumulated external system input vector and the accumulated external system output vector in the time interval $I_k=[t_k,t_{k+1}]$, $k=0,1,\ldots,n-1$, respectively.
    Furthermore, $F^k_{ij}$ denotes the accumulated flux from compartment $j$ to compartment $i$ during $I_k$.

    We are interested in constructing a discrete-time compartmental system \eqref{eqn:CS_discrete} that matches these given data as well as possible.
    To that end, we define $\widehat{\vec{u}}(k)=\vec{u}^k$, $k=0,1,\ldots,n-1$.
    Furthermore, for each time step $k=0,1,\ldots,n-1$, we have to find $\widehat{\tens{B}}(k)$ such that $\widehat{B}_{ij}(k)\,\widehat{C}_j(k)=F^k_{ij}$ and $\widehat{z}_j(k)\,\widehat{C}_j(k)=r^k_j$.
    Consequently, for $i\neq j$,
    \begin{equation*}
        \widehat{B}_{ij}(k) =
        \begin{cases}
            F^k_{ij} / \widehat{C}_j(k),\quad&\text{ if }\widehat{C}_j(k)\neq0,\\
            0, &\text{ if }\widehat{C}_j(k)=0,
        \end{cases}
    \end{equation*}
    and
    \begin{equation*}
        \widehat{B}_{jj}(k) = 
        \begin{cases}
            1-\left[\suml_{i\neq j}\widehat{B}_{ij}(k)+r^k_j/\widehat{C}_j(k)\right],\quad &\text{ if }\widehat{C}_j(k)\neq0,\\
            1, &\text{ if }\widehat{C}_j(k)=0.
        \end{cases}
    \end{equation*}
    We define $\widehat{\vec{C}}(k+1)=\widehat{\tens{B}}(k)\,\widehat{\vec{C}}(k)+\widehat{\vec{u}}(k)$ and continue with the step $k+1$.
    
    To  create a discrete-time approximation of the underlying model for arbitrary temporal resolution model output, we have two possibilities:
    \begin{enumerate}[(1)]
        \item We reconstruct a continuous-time approximation and produce discrete-time data out of the continuous-time reconstruction at the desired temporal resolution.
        \item Each time we identify a single time-step $k_0$ as too large, we construct a continuous-time approximation only on the interval $I_{k_0}$ and use it to produce additional data at times $t_{k_0}=t_{k_0}^0<t_{k_0}^1<\ldots<t_{k_0}^m=t_{k_0+1}$.
        Then we continue the discrete-time approximation based on the additional data.
        Note that this necessarily leads to a discrete-time compartmental system with variable time step, which should not cause major theoretical issues.
    \end{enumerate}


\section*{ELMv1-ECA Model Description and analysis approach (Holger, Qing, Bill)}
\begin{itemize}
    \item describe ELMv1-ECA (Zhu et al. in revision), including soil and plant CNP model, allocation, and competition. Describe soil CNP pool structure (Koven et al)
    
    The recent development of ELMv1-ECA (Zhu et al. in revision) is based on CLM4.5BGC [Koven et al., 2013], which represents vertically-resolved soil C and N dynamics based on the Century model [Parton et al., 1993] with enhancements  to include soil O$_2$ effects. ELMv1-ECA added several features that affect C inputs to soil and SOC dynamics. Briefly, the changes relevant to the current study include: (a) a prognostic phosphorus cycle based on Wang et al. [2007]; (b) an Equilibrium Chemistry Approximation [Tang and Riley, 2013; Zhu et al., 2017; Zhu et al., 2016]  approach to represent nutrient competition between plants, microbes, and mineral surfaces; (c) dynamic leaf stoichiometry that affects photosynthesis [Ghimire et al., 2016; Walker et al., 2014]; and (d) dynamic C, N, and P allocation within the plant based on Friedlingstein et al. [1999], which considers light, water, soil nitrogen and phosphorus stresses. ELMv1-ECA has been evaluated against multiple global-scale observations of ecosystem C, water, and energy stocks and fluxes using ILAMB (Collier et al. 2018; Zhu et al. in revision) and short-term nutrient cycling observations and global-scale partitioning of N losses (Riley et al. 2018). Overall, model benchmarking shows improvement from the precursor CLM4.5 predictions, and in particular for this study, more accurate estimates of spatially-distributed soil C stocks.
    
    \item Present the simulation protocol (spinup, forcing, site characteristics) and how 14C is calculated.
    
    Characteristics for the three sites were extracted from the standard global model datasets, which are used to set (pfts, soil properties, etc.). The simulation protocol for all three sites follows the standard approach described in Oleson et al. (2013). The model is first spun up (i.e., brought close to equilibrium) using an accelerated soil decomposition approach (Koven et al., 2013) for 1000 years, followed by a 200 year regular spin up with regular soil decomposition. Soil phosphorus pools were initialized from observations (Yang et al., 2013) at the beginning of the regular spinup. The spinup simulations were forced with repeated meteorology and constant atmospheric CO$_2$ mole fraction (285 ppm). We evaluate the spun-up model's net carbon budget to ensure changes at the end of the spin up are less than X percent/year. After spinup, the model was run in a transient simulation from 1850 to 2010 with GSWP reanalysis forcing (Dirmeyer et al., 2006), transient CO$_2$ concentrations, N deposition (Lamarque et al., 2005), and P deposition (Mahowald et al., 2008). 
    
    
    \item Show different ways to aggregate the model output, e.g. weighted radiocarbon values for single pools across depth layers or all pools for each layer
    
    We performed analyses using several approaches to aggregate ELMv1-ECA predictions. First, to characterize the effect of temporal discretization, we applied the Holgerization approach (we need a name for this approach) using ELMv1-ECA output averaged over 1, 10, and 30 d time steps \red{we did not do 30d, it might even end up quite complicated to do}. This analysis will inform required output frequency for other models to apply Holgerization. We also tested the effects of consolidating all soil C pool states and fluxes at each depth layer on the predicted $\Delta^{14}$C values of the total soil C stock. 
    \item \gray{Show predictions of pool age, system age, and transit time predicted by our approach}
\end{itemize}    

\section{Results}

\subsection{1D compartmental system example}
        
\textbf{Example.}
Consider the one-dimensional compartmental system
\begin{equation}\label{eqn:CS_one_dim_example}
    \begin{aligned}
        \deriv{t}\,C(t) &= -0.04\,C(t) + \left[3+\sin(t/20)\right],\quad t>1909,\\
        C(1909) &= 40.
    \end{aligned}
\end{equation}
Figure \ref{fig:CS_one_dim_example} depicts the quality of approximation with respect to a different number $n$ of given data points.
\begin{figure}[htbp]
    \centering 
    \includegraphics[width=1.0\linewidth]{figs/interpol_pwc_1.pdf}
    \caption{C stocks over time.
        The solid black line represents the carbon content of the original system \eqref{eqn:CS_one_dim_example}, the dashed red line represents the carbon content of the approximating system.
        Different panels represent a different number of given equidistant data points, represented by vertical lines.
        }
    \label{fig:CS_one_dim_example}
\end{figure}        

\textbf{Example.}
Consider the one-dimensional compartmental system
\begin{equation}\label{eqn:CS_one_dim_example_auton}
    \begin{aligned}
        \deriv{t}\,C(t) &= -0.04\,C(t) + 3,\quad t>1909,\\
        C(1909) &= 40.
    \end{aligned}
\end{equation}
Figure \ref{fig:CS_one_dim_example_auton} shows that the approximation is perfect for $n=3,6$.
\begin{figure}[htbp]
    \centering 
    \includegraphics[width=1.0\linewidth]{figs/interpol_pwc_1_auton.pdf}
    \caption{C stocks over time.
        The solid black line represents the carbon content of the original system \eqref{eqn:CS_one_dim_example_auton}, the dashed red line represents the carbon content of the approximating system.
        Different panels represent a different number of given equidistant data points, represented by vertical lines.
        }
    \label{fig:CS_one_dim_example_auton}
\end{figure}        

\subsection{ELM Results}
Figures:

- demonstration of approach working for different time steps (1 vs 10 vs 30 d) for 12C
-     same for 14C
- vertical structure
- lumping SOC pools by layer, and lumping those lumped SOC by depth interval (e.g., 0-30 cm)

\section*{Discussion (Bill, Alison)}

\subsection{Performance and future application }
This approach can be applied to both single-depth and vertically-resolved models run at different spatial and temporal resolutions. Using a small time-step (1 and 10 days) we are able to match modeled 14C output with extraordinary precision ($<$0.000XX\% relative error). Although the error rate increases for larger time steps, reasonable error ($<$0.0X\% relative error) is still achieved at the 30-day time step. As this is the standard output time step from most earth system models, this demonstrates the widespread applicability of this approach. 

This approach can be used across a range of models, as it does not require models to implement additional isotope dynamics or tracers, but simply to output existing stocks and fluxes of total C. For maximum accuracy, all model stocks and fluxes are required (e.g. all individual transfers between all pools). However, the approach can also be applied to aggregated C stocks (eg total soil C instead of individual pools). This reduces the accuracy of the model representation, but a test with ESM showed.... We did this and found XX increased error...that is still within the range of value given the uncertainties inherent to Earth system models...

\red{Make sure to cover: (1) what is required to make the approach work (fluxes and states that affect 12C); application to existing models; (2) application to single- and multi-layer models; (3) does the approach work for total C in each layer (i.e., if you aggregate all C pools in each layer). We could aggregate all fluxes in and all fluxes out of a soil layer}

\subsection{Potential for data-model comparison}

The potential of this approach to add radiocarbon output across a range of models will allow valuable data-model comparisons. This work will enable much more widespread use of radiocarbon as an additional model constraint, particularly in soils. It provides a framework for model intercomparison and will enable the future use of radiocarbon datasets to benchmark and calibrate earth system models.

Although radiocarbon is commonly associated with C age, it cannot be directly interpreted as such in soils, due to mixing of C inputs of different ages in the open system. Instead, radiocarbon measurements and model output are primarily of value as a tracer, and for direct model-data comparison. However, by using a state transition matrix approach, there is the added possibility to calculate ages and transit times from the system (Metzler et al, 2018). In the future, this approach could be applied to compute radiocarbon values, ages and transit times across a suite of models. This would allow not only direct data-model validation, but also meaningful conceptual comparisons of the age and transit time of C across model structures. 


\section*{Acknowledgements}
Funding was provided by the Max Planck Society and the German Research Foundation through its Emmy Noether Program (SI 1953/2--1). A.M.H. received funding from the European Research Council (ERC) under the European Union’s Horizon 2020 research and innovation programme (grant agreement No. 695101 (14Constraint)).   

\bibliographystyle{apalike}
%\bibliographystyle{gcb}
\bibliography{refs}


\end{document}